\documentclass{article}

\usepackage{graphicx}
\usepackage{color}
\usepackage{array}
\usepackage{longtable}
\usepackage{comment} 
\usepackage{hyperref}

\begin{document}

\title{jStar Eclipse tutorial} 
\maketitle 


\section{Installation}

Prerequisites:
\begin{itemize}
   \item JDK 6 (not JRE since it does not have a compiler needed for annotation processing). You can specify it in eclipse.ini:\\\\
Windows Example\\
-vm\\
C:\textbackslash Java\textbackslash JDK\textbackslash 1.6\textbackslash bin\textbackslash javaw.exe\\

Linux Example\\
-vm\\
/opt/sun-jdk-1.6.0.02/bin/java\\

Mac Example\\
-vm\\
/System/Library/Frameworks/JavaVM.framework/Versions/1.6.0/Home/bin/java\\

For more information go to \href{http://wiki.eclipse.org/Eclipse.ini}{http://wiki.eclipse.org/Eclipse.ini}

\end{itemize}
Two ways:
\begin{enumerate}
\item Update site (Recommended). Go to {\bf  Help} $\rightarrow$ {\bf Install New Software...}. Click {\bf Add...}  and type in the name and update site for the plugin. \\ The update site for the jStar Eclipse Plug-in is \\ {\bf http://github.com/da319/jStar-eclipse/raw/master/com.jstar.eclipse.update.site/}\\
Press {\bf OK}.\\
Uncheck {\bf Group items by category}.\\
\includegraphics[width=4in]{images/updateSite1.jpg}\\\\
The jStar Eclipse Plug-in should appear, as the following figure shows.\\\\
\includegraphics[width=4in]{images/updateSite2.jpg}\\\\
Select the plugin and click {\bf Next}.\\
Click {\bf Next} in the following dialog.
Accept the terms of the license agreement and click {\bf Finish}.
\item Manually. Add the latest plugin jar file\\ \textbf{com.jstar.eclipse.update.site/plugins/com.jstar.eclipse\_1.0.0.x.jar} to \texttt{eclipse/dropins/} folder and restart Eclipse.
\end{enumerate}

\section{Configuration}

\subsection{Windows}

Go to Window $\rightarrow$ Preferences $\rightarrow$ jStar and set the required directories.\\

\includegraphics[width=4in]{images/preferences.jpg}\\

rt.jar could be found in jdk1.6.0/jre/lib/. It is required library for soot. The plugin will try to find it automatically. 

\subsection{Linux}
rt.jar could be found in jdk1.6.0/jre/lib/. It is required library for soot. The plugin will try to find it automatically. 

\subsection{Mac}
classes.jar and ui.jar could be found in \\ /System/Library/Frameworks/JavaVM.framework/Versions/1.6.0/Classes/.  They are required libraries for soot. The plugin will try to find them automatically. 

\section {Structure}

\subsection{jStar root folder}
\label{sec:jstarrootfolder}

All jStar input files (specification, logic rules, abstraction rules, generated jimple files) are stored in jStar root folder. To specify jStar root folder open project context menu, select {\bf Properties} $\rightarrow$ {\bf jStar}.\\

\includegraphics[width=4in]{images/rootFolder.jpg}

\section{Verification}
You can verify source file by selecting \texttt{Verify with jStar} from the context menu or the main toolbar. 

By selecting \texttt{Verify with jStar Configurations...}, you can indicate specifications, logic and abstraction rules.\\ 

\includegraphics[width=4in]{images/verificationWindow.jpg}\\

In the window above you can create a new empty file or import a file for specification, logic and abstraction rules. The location depends on the location of the source file and could not be changed. Files are  are contained in jStar root folder \ref{sec:jstarrootfolder} and locations shown are relative to the project.

\subsection*{Specification in source file}

If you want to write specification in the source file, you need to add annotations.jar (can be found in com.jstar.eclipse.annotations / jar file) to your project Java Build Path. More information about annotations could be found in \ref{sec:annotationprocessing} .

\subsection* {Verification errors}

In case there are some verification errors, you can see error messages in console. The lines in source code where the problem appeared are annotated as squiggly marks.\\

\includegraphics[width=4in]{images/console.jpg}\\

\includegraphics[width=2in]{images/marker.jpg}

\section{Annotation Processing}
\label{sec:annotationprocessing}

\subsection*{Annotations}

\begin{itemize}
\item \texttt{@Import} has one element \texttt{String[] value}. You need to write a name of the class (including package declaration) which specification you want to import. This annotation can be used to annotate only type declarations.
\item \texttt{@Predicate} is used for \texttt{define} and \texttt{export} statements. It has three elements: \texttt{String predicate}, \texttt{String formula}, \texttt{DefinitionType type}. \texttt{DefinitionType} is enum with two values \texttt{Define} and \texttt{Export}.  The default value of type is \texttt{DefinitionType.Define}. This annotation can be used to annotate only type declarations.
\item \texttt{@Predicates} is used if you want to have more than one  \texttt{define} and/or \texttt{export} statements. It has one element \texttt{Predicate[] value}. This annotation can be used to annotate only type declarations.
\item \texttt{@InitSpec} is a (dynamic / both static and dynamic) specification for constructor which is not explicitly defined in the source code. It has two elements: \texttt{String pre}, \texttt{String post}. This annotation can be used to annotate only type declarations.
\item \texttt{@InitSpecs} is (dynamic / both static and dynamic) specifications which are conjuncted with \texttt{also} for constructor which is not explicitly defined in the source code. It has one element \texttt{InitSpec[] value}. This annotation can be used to annotate only type declarations.
\item \texttt{@InitSpecStatic} is a static specification for constructor which is not explicitly defined in the source code. It has two elements: \texttt{String pre}, \texttt{String post}. This annotation can be used to annotate only type declarations.
\item \texttt{@InitSpecsStatic} is static specifications which are conjuncted with \texttt{also} for constructor which is not explicitly defined in the source code. It has one element \texttt{InitSpecStatic[] value}. This annotation can be used to annotate only type declarations.
\item \texttt{@Spec} is a (dynamic / both static and dynamic) specification for a method or a constructor. It has two elements: \texttt{String pre}, \texttt{String post}. This annotation can be used to annotate only method and constructor declarations.
\item \texttt{@Specs} is (dynamic / both static and dynamic) specifications which are conjuncted with \texttt{also} for a method or a constructor. It has one element: \texttt{Spec[] value}. This annotation can be used to annotate only method and constructor declarations.
\item \texttt{@SpecStatic} is a static specification for a method or a constructor. It has two elements: \texttt{String pre}, \texttt{String post}. This annotation can be used to annotate only method and constructor declarations.
\item \texttt{@SpecsStatic} is static specifications which are conjuncted with \texttt{also} for a method or a constructor. It has one element: \texttt{SpecStatic[] value}. This annotation can be used to annotate only method and constructor declarations.
\end{itemize}


Examples of annotations in the source code:\\
\begin{longtable}{ m{7cm} | m{5cm} }
Annotation in source file & Generated specification file \\
\hline
\begin{verbatim}
@Import("java.lang.Object")
\end{verbatim}
& 
\begin{verbatim}
import("java/lang/Object.spec");
\end{verbatim}
\\
\begin{verbatim}
@Import({"Spec", "java.lang.Object"})
\end{verbatim}
& 
\begin{verbatim}
import("Spec.spec");
import("java/lang/Object.spec");
\end{verbatim}
\\
\begin{verbatim}
@Predicate(
   predicate = "P(x)", 
   formula = "F(x)"
)
\end{verbatim} 
&
\begin{verbatim}
define P(x) as F(x);
\end{verbatim}
\\
\begin{verbatim}
@Predicate(
   predicate = "P(x)", 
   formula = "F(x)", 
   type = DefinitionType.Export
)
\end{verbatim}  
&
\begin{verbatim}
export P(x) as F(x);
\end{verbatim}\\
\begin{verbatim}
@Predicates({
   @Predicate(
      predicate = "P1(x)", 
      formula = "F1(x)", 
      type = DefinitionType.Export
   ),
   @Predicate(
      predicate = "P2(x)", 
      formula = "F2(x)"
   )
})
\end{verbatim} 
& 
\begin{verbatim}
export P1(x) as F1(x);
define P2(x) as F2(x);
\end{verbatim}
\\
\begin{verbatim}
@InitSpec(
   pre = "precondition", 
   post = "postcondition"
)
\end{verbatim}
&
\begin{verbatim}
void <init>() :
   { precondition }
   { postcondition }
\end{verbatim}
\\
\begin{verbatim}
@InitSpecs({
   @InitSpec(
      pre = "precondition 1", 
      post = "postcondition 1"
   ),
   @InitSpec(
      pre = "precondition 2",
      post = "postcondition 2"
   )
})
\end{verbatim}
&
\begin{verbatim}
void <init>() :
   { precondition 1 }
   { postcondition 1 }
   andalso
   { precondition 2 }
   { postcondition 2 }
\end{verbatim}
\\
\begin{verbatim}
@InitSpecStatic(
   pre = "precondition",
   post = "postcondition"
)
\end{verbatim}
&
\begin{verbatim}
void <init>() static :
   { precondition }
   { postcondition }
\end{verbatim}
\\
\begin{verbatim}
@InitSpecsStatic({
   @InitSpecStatic(
      pre = "precondition 1", 
      post = "postcondition 1"
   ),
   @InitSpecStatic(
      pre = "precondition 2",
      post = "postcondition 2"
   )
})
\end{verbatim}
&
\begin{verbatim}
void <init>() static :
   { precondition 1 }
   { postcondition 1 }
   andalso
   { precondition 2 }
   { postcondition 2 }
\end{verbatim}
\\
\begin{verbatim}
@Spec(
   pre = "precondition", 
   post = "postcondition"
)
\end{verbatim}
\it{method declaration}
&
{\it method declaration} \texttt{:}
\begin{verbatim}
   { precondition }
   { postcondition }
\end{verbatim}
\\
\begin{verbatim}
@Specs({
   @Spec(
      pre = "precondition 1", 
      post = "postcondition 1"
   ),
   @Spec(
      pre = "precondition 2", 
      post = "postcondition 2"
   )
})
\end{verbatim}
\it{method declaration}
&
{\it method declaration} \texttt{:}
\begin{verbatim}
   { precondition 1 }
   { postcondition 1 }
   andalso
   { precondition 2 }
   { postcondition 2 }
\end{verbatim}
\\
\begin{verbatim}
@SpecStatic(
   pre = "precondition", 
   post = "postcondition"
)
\end{verbatim}
\it{method declaration}
&
{\it method declaration} \texttt{static :}
\begin{verbatim}
   { precondition }
   { postcondition }
\end{verbatim}
\\
\begin{verbatim}
@SpecsStatic({
   @SpecStatic(
      pre = "precondition 1", 
      post = "postcondition 1"
   ),
   @SpecStatic(
      pre = "precondition 2", 
      post = "postcondition 2"
   )
})
\end{verbatim}
\it{method declaration}
&
{\it method declaration} \texttt{static :}
\begin{verbatim}
   { precondition 1 }
   { postcondition 1 }
   andalso
   { precondition 2 }
   { postcondition 2 }
\end{verbatim}
\end{longtable}

\subsection*{Eclipse plugin}

You can write specifications in your java source file. By selecting an option ``Specification is included in the source file" in verification configuration, the specification file will be generated from your annotations. This specification file will be used to verify your program.

\subsection*{Command line}

You can generate a specification file from your java source file with annotations using command line:\\
{\bf javac}\\
\hspace*{20 pt}{\bf-proc:only} \\
\hspace*{20 pt}{\bf -cp} ``.;jstar\_processing.jar;commons-io-1.4.jar;commons-lang-2.5.jar;annotations.jar" \\
\hspace*{20 pt}{\bf -d} . \\
\hspace*{20 pt}{\bf -processor} com.jstar.eclipse.processing.SpecAnnotationProcessor \\
\hspace*{20 pt}{\it MyClass.java}\\\\
{\bf -proc:only }\\
\hspace*{20pt} Only annotation processing is done, without any subsequent compilation.\\
{\bf -cp}\\
\hspace*{20pt} Specifies the classpath. As well as specifying the classpath to the classes MyClass is referencing to, you need to add the following jar files:
\begin{itemize}
\item jstar\_processing.jar \\ where the processor {\bf com.jstar.eclipse.processing.SpecAnnotationProcessor} lives in.
\item commons-io-1.4.jar, commons-lang-2.5.jar \\ required libraries for annotation processing.
\item annotations.jar \\ where the specification annotations live in.
\end{itemize}
{\bf -d} \\ 
\hspace*{20 pt}The destination directory for the specification file.\\
{\bf -processor}\\
\hspace*{20 pt}The name of annotation processor. \\
{\it MyClass.java}\\
\hspace*{20 pt} One or more source files with annotations.
\end{document}